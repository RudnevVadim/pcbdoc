
\section{Спецификация}

Исходный код спецификации, как и любого другого документа \LaTeX{}, должен
начинаться с команды \bfemph{\textbackslash{}documentclass}. В основном аргументе
команды(в фигурных скобках) следует указать класс документа \sfemph{pcbdoc}, а в
необязательном(в квадратных скобках) --- ключ \sfemph{doctype=spec}.

\begin{pcbdoccode}
\documentclass[doctype=spec]{pcbdoc}
\end{pcbdoccode}

Далее(в преамбуле) должны находиться команды заполнения полей документа. Их можно
разместить либо непосредственно в преамбуле, либо в отдельном пакете, включаемом в
преамбулу с помощью команды \bfsf{\textbackslash{}usepackage}. Тело документа,
находящееся после преамбулы, должно начинаться с команды

\begin{pcbdoccode}
\begin{document}
\end{pcbdoccode}

а заканчиваться командой

\begin{pcbdoccode}
\end{document}
\end{pcbdoccode}

Внутри тела документа должны находиться команды заполнения строк спецификации.

Базовой командой заполнения строки спецификации, на которой основаны все ос\-таль\-ные
команды, является команда \bfemph{\textbackslash{}Line}. Её необходимо использовать,
если функциональности основанных на ней команд недостаточно. Эта команда принимает семь
аргументов, которые печатаются в колонках
\resultbox{\sfemph{~~Формат~~}}, \resultbox{\sfemph{~~Зона~~}},
\resultbox{\sfemph{~~Поз.~~}}, \resultbox{\sfemph{Обозначение}},
\resultbox{\sfemph{Наименование}}, \resultbox{\sfemph{Кол.}} и
\resultbox{\sfemph{Примечание}}. Команды, основанные на
команде \bfemph{\textbackslash{}Line}, могут иметь встроенный счётчик, значение которого
заносится в колонку \resultbox{\sfemph{Поз.}}. Описание команды
\bfemph{\textbackslash{}Line}, а также команд заполнения строк спецификации без
встроенного счётчика приведенно в \bfsf{Таблице~\ref{tabular:speclines1}}.\sloppy

\fussy
Базовой командой заполнения строки спецификации со встроенным счётчиком является команда
\bfemph{\textbackslash{}Detail}. Она основана на команде \bfemph{\textbackslash{}Line} и
принимает семь аргументов, которые печатаются в колонках
\resultbox{\sfemph{Поз.}}, \resultbox{\sfemph{Формат}},
\resultbox{\sfemph{Зона}}, \resultbox{\sfemph{Обозначение}},
\resultbox{\sfemph{Наименование}}, \resultbox{\sfemph{Кол.}} и
\resultbox{\sfemph{Примечание}}. Как и в случае с командой
\bfemph{\textbackslash{}Line}, данную команду имеет смысл использовать при отсутствии
требуемой функциональности у основанных на ней команд. Описание команды
\bfemph{\textbackslash{}Detail}, а также команд заполнения строк спецификации со
встроенным счётчиком приведено в \bfsf{Таблице~\ref{tabular:speclines2}}.\sloppy

\fussy
\clearpage

\tikztableset{speclines table1}{
  draw,
  row 1/.style={minimum height=6mm,font=\sffamily\bfseries,align=center},
  column 1/.style={font=\sffamily\bfseries\itshape\small,align=left,text width=26mm},
  column 2/.style={font=\small,align=justify,text width=117mm}
}

\begin{tikztablex}[speclines table1]
{
\caption{Команды заполнения строк спецификации\\без встроенного счётчика}
\label{tabular:speclines1}
}
{
%%%
Команда & Описание\\
%%%
{\textbackslash{}Line\\
\{<format>\}\\
\{<zone>\}\\
\{<reference>\}\\
\{<designating>\}\\
\{<naming>\}\\
\{<quantity>\}\\
\{<note>\}}
&
Базовая команда заполнения строки спецификации. Аргументы \sfemph{<format>},
\sfemph{<zone>}, \sfemph{<reference>}, \sfemph{<designating>}, \sfemph{<naming>},
\sfemph{<quantity>} и \sfemph{<note>} печатаются в колонках
\resultbox{\sfemph{~~Формат~~}}, \resultbox{\sfemph{~~Зона~~}},
\resultbox{\sfemph{~~Поз.~~}}, \resultbox{\sfemph{Обозначение}},
\resultbox{\sfemph{Наименование}}, \resultbox{\sfemph{Кол.}} и
\resultbox{\sfemph{Примечание}} соответственно.\\
%%%
\textbackslash{}Part\{<name>\} &
Печатает подчёркнутый текст \sfemph{<name>} в центре колонки
\resultbox{\sfemph{Наименование}}\\
%%%
\textbackslash{}VariableData &
Печатает подчёркнутый текст \sfemph{Переменные данные для исполнений} в колонках
\resultbox{\sfemph{Обозначение}} и \resultbox{\sfemph{Наименование}}.\\
%%%
{\textbackslash{}VariantNumber\\[0pt][<number>]} &
Печатает подчёркнутый текст вида \sfemph{<NUMBER>-<number>} в колонке
\resultbox{\sfemph{Наименование}}, где \sfemph{<NUMBER>} - аргумент команды
\bfemph{\textbackslash{}NumberSet}(см. таблицу~\ref{tabular:fields}). Наличие аргумента
\sfemph<number> необязательно.\\
%%%
\textbackslash{}Sb\{<format>\} &
Заполняет строку спецификации, добавляя запись о сборочном чертеже. Аргумент
\sfemph{<format>} печатается в колонке \resultbox{\sfemph{Формат}}. В
случае, если аргумент \sfemph{<format>} не помещается в этой колонке, он печатается в
колонке \resultbox{\sfemph{Примечание}}, а в колонке
\resultbox{\sfemph{Формат}} печатается символ \bfemph{*}. В колонке
\resultbox{\sfemph{Обозначение}} печатается текст вида \sfemph{<number>СБ},
где \sfemph{<number>} - аргумент команды
\bfemph{\textbackslash{}NumberSet}(см. таблицу~\ref{tabular:fields}). В колонке
\resultbox{\sfemph{Наименование}} печатается текст \sfemph{Сборочный
чертеж}.\\
%%%
\textbackslash{}Sch\{<format>\} &
Заполняет строку спецификации, добавляя запись о схеме электричес\-кой принципиальной.
Аргумент \sfemph{<format>} печатается в колонке
\resultbox{\sfemph{Формат}}. В случае, если аргумент \sfemph{<format>} не помещается в
этой колонке, он пе\-ча\-та\-ет\-ся в колонке
\resultbox{\sfemph{Примечание}}, а в колонке
\resultbox{\sfemph{Формат}} пе\-ча\-та\-ет\-ся символ \bfemph{*}. В колонке
\resultbox{\sfemph{Обозначение}} печатается текст вида \sfemph{<number>Э3},
где \sfemph{<number>} - аргумент команды
\bfemph{\textbackslash{}NumberSet}(см. таб\-ли\-цу~\ref{tabular:fields}). В колонке
\resultbox{\sfemph{Наименование}} печатается двустрочный текст
\sfemph{Схема электрическая принципиальная}.\looseness=-1\\

%%%
\textbackslash{}El &
Заполняет строку спецификации, добавляя запись о перечне элементов.
В колонке \resultbox{\sfemph{Формат}} печатается текст \sfemph{A4}.
В колонке \resultbox{\sfemph{Обозначение}} печатается текст вида
\sfemph{<number>ПЭ}, где \sfemph{<number>} - аргумент команды
\bfemph{\textbackslash{}NumberSet}(см. таблицу~\ref{tabular:fields}). В колонке
\resultbox{\sfemph{Наименование}} печатается текст
\sfemph{Перечень элементов}.\\
};
\coordinate(top) at (tikztable-1-2.north);
\coordinate(left) at (tikztable-1-1.west);
\coordinate(right) at (tikztable-1-2.east);
\draw[line width=0.6mm] (top-|left) -- (top-|right);

\coordinate(top) at (tikztable-1-2.south);
\draw[line width=0.6mm] (top-|left) -- (top-|right);

\coordinate(top) at (tikztable-3-2.north);
\draw (top-|left) -- (top-|right);

\coordinate(top) at (tikztable-4-2.north);
\draw (top-|left) -- (top-|right);

\coordinate(top) at (tikztable-5-2.north);
\draw (top-|left) -- (top-|right);

\coordinate(top) at (tikztable-6-2.north);
\draw (top-|left) -- (top-|right);

\coordinate(top) at (tikztable-7-2.north);
\draw (top-|left) -- (top-|right);

\coordinate(top) at (tikztable-8-2.north);
\draw (top-|left) -- (top-|right);

\coordinate(top) at (tikztable-8-2.south);
\draw[line width = 0.6mm] (top-|left) -- (top-|right);

\begin{scope}[on background layer]
\coordinate(top) at (tikztable-2-1.north);
\coordinate(bottom) at (tikztable-8-2.south);
\coordinate(left) at (tikztable-1-1.west);
\coordinate(right) at (tikztable-1-1.east);
\fill[codecolor] (top-|left) rectangle (bottom-|right);
\end{scope}
\end{tikztablex}

\clearpage

\begin{tikztablex}[draw,speclines table1]
{
\caption*{Таблица~\ref{tabular:speclines1}. Команды заполнения строк спецификации\\без
встроенного счётчика. Продолжение}
}
{
%%%
Команда & Описание\\
%%%
\textbackslash{}Dd[<note>] &
Заполняет строку спецификации, добавляя запись о конструкторских данных. В колонке
\resultbox{\sfemph{Формат}} печатается символ \sfemph{-}. В колонке
\resultbox{\sfemph{Обозначение}} печатается текст вида
\sfemph{<number>Д36}, где \sfemph{<number>} - аргумент команды
\bfemph{\textbackslash{}PcbNumberSet}(см. таблицу~\ref{tabular:fields}). В колонке
\resultbox{\sfemph{Наименование}} печатается текст \sfemph{Конструкторские
данные}. В колонке \resultbox{\sfemph{Примечание}} печатается
необязательный аргумент \sfemph{<note>}(\sfemph{на CD} по умолчанию)\\
%%%
\textbackslash{}ICd[<note>] &
Заполняет строку спецификации, добавляя запись о данных микросхем. В колонке
\resultbox{\sfemph{Формат}} печатается символ \sfemph{-}. В колонке
\resultbox{\sfemph{Обозначение}} печатается текст вида
\sfemph{<number>Д66}, где \sfemph{<number>} - аргумент команды
\bfemph{\textbackslash{}NumberSet}(см. таблицу~\ref{tabular:fields}). В колонке
\resultbox{\sfemph{Наименование}} печатается текст \sfemph{Данные
микросхем}. В колонке \resultbox{\sfemph{Примечание}} печатается
необязательный аргумент \sfemph{<note>}(\sfemph{на CD} по умолчанию)\\
%%%
\textbackslash{}DigDoc[<note>] &
Заполняет строку спецификации, добавляя запись о документации в электронном виде. В
колонке \resultbox{\sfemph{Формат}} печатается символ \sfemph{-}. В колонке
\resultbox{\sfemph{Обозначение}} печатается текст вида
\sfemph{<number>ДМ}, где \sfemph{<number>} - аргумент команды
\bfemph{\textbackslash{}NumberSet}(см. таблицу~\ref{tabular:fields}). В колонке
\resultbox{\sfemph{Наименование}} печатается двустрочный текст
\sfemph{КД на магнитном носителе данных}. В колонке
\resultbox{\sfemph{Примечание}} печатается необязательный аргумент
\sfemph{<note>}(\sfemph{на CD} по умолчанию)\\
%%%
\textbackslash{}Ai &
Заполняет строку спецификации, добавляя запись о инструкции по настройке. В колонке
\resultbox{\sfemph{~Формат~}} печатается текст \sfemph{A4}. В колонке
\resultbox{\sfemph{Обозначение}} печатается текст вида
\sfemph{<number>И2}, где \sfemph{<number>} - аргумент команды
\bfemph{\textbackslash{}NumberSet}(см. таблицу~\ref{tabular:fields}). В колонке
\resultbox{\sfemph{Наименование}} печатается текст
\sfemph{Инструкция по настройке}.\\
};
\coordinate(top) at (tikztable-1-2.north);
\coordinate(left) at (tikztable-1-1.west);
\coordinate(right) at (tikztable-1-2.east);
\draw[line width=0.6mm] (top-|left) -- (top-|right);

\coordinate(top) at (tikztable-1-2.south);
\draw[line width=0.6mm] (top-|left) -- (top-|right);

\coordinate(top) at (tikztable-3-2.north);
\draw (top-|left) -- (top-|right);

\coordinate(top) at (tikztable-4-2.north);
\draw (top-|left) -- (top-|right);

\coordinate(top) at (tikztable-5-2.north);
\draw (top-|left) -- (top-|right);

\coordinate(top) at (tikztable-5-2.south);
\draw[line width=0.6mm] (top-|left) -- (top-|right);
\begin{scope}[on background layer]
\coordinate(top) at (tikztable-2-2.north);
\coordinate(bottom) at (tikztable-5-2.south);
\coordinate(left) at (tikztable-1-1.west);
\coordinate(right) at (tikztable-5-1.east);
\fill[codecolor] (top-|left) rectangle (bottom-|right);
\end{scope}
\end{tikztablex}

Команды заполнения строк без встроенного счётчика используются для
создания разделов спецификации(команда \bfemph{\textbackslash{}Part}), а также для
заполнения раздела \sfemph{Документация}. Например:

\begin{pcbdoccode}
...
\Part{Документация}
% Сборочный чертеж:
\Sb{A3}
% Схема электрическая принципиальная:
\Sch{A3}
% Перечень элементов:
\El
% Инструкция по настройке:
\Ai
% Данные микросхем:
\ICd
% Конструкторские данные:
\Dd
% Документация в электронном виде:
\DigDoc
...
\end{pcbdoccode}

\tikztableset{speclines table2}{
  draw,
  %nodes=draw,
  row 1/.style={minimum height=6mm,font=\sffamily\bfseries,align=center},
  column 1/.style={font=\sffamily\bfseries\itshape\small,align=left,text width=35mm},
  column 2/.style={font=\small,align=justify,text width=105mm},
}

\begin{tikztablex}[draw,speclines table2]
{
\caption{Команды заполнения строк спецификации\\со встроенным счётчиком}
\label{tabular:speclines2}
}
{
%%%
Команда & Описание\\
%%%
{\textbackslash{}Detail[<reference>]\\
\{<format>\}\\
\{<zone>\}\\
\{<designating>\}\\
\{<naming>\}\\
\{<quantity>\}\\
\{<note>\}}
& Базовая команда заполнения строки спецификации со встроенным
счётчиком. Необязательный аргумент \sfemph{<reference>} управляет встроенным счётчиком
позиционного обозначения. Если аргумент \sfemph{<reference>} имеет значение \sfemph{-},
счетчик позиционного обозначения не инкрементируется, а в колонку
\resultbox{\sfemph{Поз.}} записывается символ \sfemph{-}. Если данный
аргумент имеет значение \sfemph{0}, счетчик позиционного обозначения также не
инкрементируется, а в колонку \resultbox{\sfemph{Поз.}} ничего не
записывается. Если значением аргумента является положительное число, счётчик
инкрементируется на это число, а если отрицательное --- счётчик принимает значение
модуля этого числа. В двух последних случаях в колонку
\resultbox{\sfemph{Поз.}} записывается значение счетчика. Значением
этого аргумента по умолчанию является \sfemph{1}. Аргументы \sfemph{<format>},
\sfemph{<zone>}, \sfemph{<designating>}, \sfemph{<naming>}, \sfemph{<quantity>} и
\sfemph{<note>} печатаются в колонках
\resultbox{\sfemph{Формат}},
\resultbox{\sfemph{Зона}},
\resultbox{\sfemph{Обозначение}},
\resultbox{\sfemph{Наименование}},
\resultbox{\sfemph{Кол.}} и
\resultbox{\sfemph{Примечание}} соответственно.\\
%%%
{\textbackslash{}Pp[<reference>]\\
\{<format>\}\\
\{<zone>\}\\
\{<note>\}}
&
Команда, производная от команды \bfemph{\textbackslash{}Detail}. Заполняет строку
спецификации, добавляя запись о печатной плате. Необязательный аргумент
\sfemph{<reference>} управляет встроенным счётчиком позиционного
обозначения(см. описание команды \bfemph{\textbackslash{}Detail}). В колонке
\resultbox{\sfemph{Обозначение}} печатается аргумент команды
\bfemph{\textbackslash{}PcbNumber}(см. таблицу~\ref{tabular:fields}). Аргументы
\sfemph{<format>}, \sfemph{<zone>} и \sfemph{<note>} печатаются в колонках
\resultbox{\sfemph{Формат}}, \resultbox{\sfemph{Зона}} и
\resultbox{\sfemph{Примечание}} соответственно. В колонке
\resultbox{\sfemph{Наименование}} печатается текст \sfemph{Плата печатная}.\\
%%%
{\textbackslash{}Component\\[0pt][<reference>]\\
\{<zone>\}\\
\{<naming>\}\\
\{<quantity>\}\\
\{<note>\}\\}
&
Команда, производная от команды \bfemph{\textbackslash{}Detail}. Заполняет строку
спецификации, добавляя запись об электронном компоненте(резистор, транзистор,
микросхема и т.д.). Необязательный аргумент \sfemph{<reference>} управляет встроенным
счётчиком позиционного обозначения(см. описание команды
\bfemph{\textbackslash{}Detail}). Аргументы \sfemph{<zone>}, \sfemph{<naming>}
\sfemph{<quantity>>} и \sfemph{<note>} печатаются в колонках \resultbox{\sfemph{Зона}},
\resultbox{\sfemph{Наименование}}, \resultbox{\sfemph{Количество}} и
\resultbox{\sfemph{Примечание}} соответственно.\\
};
\coordinate(top) at (tikztable-1-2.north);
\coordinate(left) at (tikztable-1-1.west);
\coordinate(right) at (tikztable-1-2.east);
\draw[line width=0.6mm] (top-|left) -- (top-|right);

\coordinate(top) at (tikztable-1-2.south);
\draw[line width=0.6mm] (top-|left) -- (top-|right);

\coordinate(top) at (tikztable-3-2.north);
\draw (top-|left) -- (top-|right);

\coordinate(top) at (tikztable-4-2.north);
\draw (top-|left) -- (top-|right);

\coordinate(top) at (tikztable-4-2.south);
\draw[line width=0.6mm] (top-|left) -- (top-|right);

\begin{scope}[on background layer]
\coordinate(top) at (tikztable-2-2.north);
\coordinate(bottom) at (tikztable-4-2.south);
\coordinate(left) at (tikztable-1-1.west);
\coordinate(right) at (tikztable-1-1.east);
\fill[codecolor] (top-|left) rectangle (bottom-|right);
\end{scope}
\end{tikztablex}

\clearpage

\begin{tikztablex}[draw,speclines table2]
{
\caption*{Таблица~\ref{tabular:speclines2}. Команды заполнения строк спецификации\\
со встроенным счётчиком. Продолжение}
}
{
%%%
Команда & Описание\\
%%%
{\textbackslash{}Capacitor\\[0pt][<reference>]\\
\{<zone>\}\\
\{<naming>\}\\
\{<quantity>\}\\
\{<note>\}\\}
& Команда, производная от команды \bfemph{\textbackslash{}Component}. Заполняет строку
спецификации, добавляя запись о конденсаторе. Аргументы идентичны команде
\bfemph{\textbackslash{}Component}. В колонке \resultbox{\sfemph{Наименование}} перед
аргументом \sfemph{<naming>} печатается текст \sfemph{Конденсатор}.\\
%%%
{\textbackslash{}Resistor\\[0pt][<reference>]\\
\{<zone>\}\\
\{<naming>\}\\
\{<quantity>\}\\
\{<note>\}\\}
&
Команда, производная от команды \bfemph{\textbackslash{}Component}. Заполняет строку
спецификации, добавляя запись о резисторе. Аргументы идентичны команде
\bfemph{\textbackslash{}Component}. В колонке \resultbox{\sfemph{Наименование}} перед
аргументом \sfemph{<naming>} печатается текст \sfemph{Резистор}.\\
{\textbackslash{}IC\\[0pt][<reference>]\\
\{<zone>\}\\
\{<naming>\}\\
\{<quantity>\}\\
\{<note>\}\\}
&
Команда, производная от команды \bfemph{\textbackslash{}Component}. Заполняет строку
спецификации, добавляя запись о микросхеме. Аргументы идентичны команде
\bfemph{\textbackslash{}Component}. В колонке \resultbox{\sfemph{Наименование}} перед
аргументом \sfemph{<naming>} печатается текст \sfemph{Микросхема}.\\
%%%
{\textbackslash{}Relay\\[0pt][<reference>]\\
\{<zone>\}\\
\{<naming>\}\\
\{<quantity>\}\\
\{<note>\}\\}
&
Команда, производная от команды \bfemph{\textbackslash{}Component}. Заполняет строку
спецификации, добавляя запись о реле. Аргументы идентичны команде
\bfemph{\textbackslash{}Component}. В колонке \resultbox{\sfemph{Наименование}} перед
аргументом \sfemph{<naming>} печатается текст \sfemph{Реле}.\\
%%%
{\textbackslash{}Inductor\\[0pt][<reference>]\\
\{<zone>\}\\
\{<naming>\}\\
\{<quantity>\}\\
\{<note>\}\\}
&
Команда, производная от команды \bfemph{\textbackslash{}Component}. Заполняет строку
спецификации, добавляя запись о дросселе. Аргументы идентичны команде
\bfemph{\textbackslash{}Component}. В колонке \resultbox{\sfemph{Наименование}} перед
аргументом \sfemph{<naming>} печатается текст \sfemph{Дроссель}.\\
%%%
{\textbackslash{}Plug\\[0pt][<reference>]\\
\{<zone>\}\\
\{<naming>\}\\
\{<quantity>\}\\
\{<note>\}\\}
&
Команда, производная от команды \bfemph{\textbackslash{}Component}. Заполняет строку
спецификации, добавляя запись о вилке. Аргументы идентичны команде
\bfemph{\textbackslash{}Component}. В колонке \resultbox{\sfemph{Наименование}} перед
аргументом \sfemph{<naming>} печатается текст \sfemph{Вилка}.\\
%%%
{\textbackslash{}Socket\\[0pt][<reference>]\\
\{<zone>\}\\
\{<naming>\}\\
\{<quantity>\}\\
\{<note>\}\\}
&
Команда, производная от команды \bfemph{\textbackslash{}Component}. Заполняет строку
спецификации, добавляя запись о розетке. Аргументы идентичны команде
\bfemph{\textbackslash{}Component}. В колонке \resultbox{\sfemph{Наименование}} перед
аргументом \sfemph{<naming>} печатается текст \sfemph{Розетка}.\\
};
\coordinate(top) at (tikztable-1-2.north);
\coordinate(left) at (tikztable-1-1.west);
\coordinate(right) at (tikztable-1-2.east);
\draw[line width=0.6mm] (top-|left) -- (top-|right);

\coordinate(top) at (tikztable-1-2.south);
\draw[line width=0.6mm] (top-|left) -- (top-|right);

\coordinate(top) at (tikztable-3-1.north);
\draw (top-|left) -- (top-|right);

\coordinate(top) at (tikztable-4-1.north);
\draw (top-|left) -- (top-|right);

\coordinate(top) at (tikztable-5-1.north);
\draw (top-|left) -- (top-|right);

\coordinate(top) at (tikztable-6-1.north);
\draw (top-|left) -- (top-|right);

\coordinate(top) at (tikztable-7-1.north);
\draw (top-|left) -- (top-|right);

\coordinate(top) at (tikztable-8-1.north);
\draw (top-|left) -- (top-|right);

\coordinate(top) at (tikztable-8-1.south);
\draw[line width=0.6mm] (top-|left) -- (top-|right);

\begin{scope}[on background layer]
\coordinate(top) at (tikztable-2-1.north);
\coordinate(bottom) at (tikztable-8-1.south);
\coordinate(left) at (tikztable-1-1.west);
\coordinate(right) at (tikztable-1-1.east);
\fill[codecolor] (top-|left) rectangle (bottom-|right);
\end{scope}
\end{tikztablex}

\clearpage

\begin{tikztablex}[draw,speclines table2]
{
\caption*{Таблица~\ref{tabular:speclines2}. Команды заполнения строк спецификации\\
со встроенным счётчиком. Продолжение}
}
{
Команда & Описание\\
%%%
{\textbackslash{}Diode\\[0pt][<reference>]\\
\{<zone>\}\\
\{<naming>\}\\
\{<quantity>\}\\
\{<note>\}\\}
&
Команда, производная от команды \bfemph{\textbackslash{}Component}. Заполняет строку
спецификации, добавляя запись о диоде. Аргументы идентичны команде
\bfemph{\textbackslash{}Component}. В колонке \resultbox{\sfemph{Наименование}} перед
аргументом \sfemph{<naming>} печатается текст \sfemph{Диод}.\\
%%%
{\textbackslash{}Transistor\\[0pt][<reference>]\\
\{<zone>\}\\
\{<naming>\}\\
\{<quantity>\}\\
\{<note>\}\\}
&
Команда, производная от команды \bfemph{\textbackslash{}Component}. Заполняет строку
спецификации, добавляя запись о транзисторе. Аргументы идентичны команде
\bfemph{\textbackslash{}Component}. В колонке \resultbox{\sfemph{Наименование}} перед
аргументом \sfemph{<naming>} печатается текст \sfemph{Транзистор}.\\
%%%
{\textbackslash{}Jumper\\[0pt][<reference>]\\
\{<zone>\}\\
\{<naming>\}\\
\{<quantity>\}\\
\{<note>\}\\}
&
Команда, производная от команды \bfemph{\textbackslash{}Component}. Заполняет строку
спецификации, добавляя запись о джампере. Аргументы идентичны команде
\bfemph{\textbackslash{}Component}. В колонке \resultbox{\sfemph{Наименование}} перед
аргументом \sfemph{<naming>} печатается текст \sfemph{Джампер}.\\
};
\coordinate(top) at (tikztable-1-2.north);
\coordinate(left) at (tikztable-1-1.west);
\coordinate(right) at (tikztable-1-2.east);
\draw[line width=0.6mm] (top-|left) -- (top-|right);

\coordinate(top) at (tikztable-1-2.south);
\draw[line width=0.6mm] (top-|left) -- (top-|right);

\coordinate(top) at (tikztable-3-1.north);
\draw (top-|left) -- (top-|right);

\coordinate(top) at (tikztable-4-1.north);
\draw (top-|left) -- (top-|right);

\coordinate(top) at (tikztable-4-1.south);
\draw[line width=0.6mm] (top-|left) -- (top-|right);

\begin{scope}[on background layer]
\coordinate(top) at (tikztable-2-1.north);
\coordinate(bottom) at (tikztable-4-1.south);
\coordinate(left) at (tikztable-1-1.west);
\coordinate(right) at (tikztable-1-1.east);
\fill[codecolor] (top-|left) rectangle (bottom-|right);
\end{scope}
\end{tikztablex}

\clearpage
