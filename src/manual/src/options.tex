
\section{Опции класса pcbdoc}

Опции класса \emph{pcbdoc} представляют собой список вида \emph{key=value},
разделённый запятыми, где \emph{key} является именем опции, а \emph{value} ---
её значением. Допускается не указывать имя опции, элемент списка в этом случае
является значением опции. Опции указываются в необязательном аргументе команды
\PCBdocCode{\textbackslash{documentclass}}.

\subsection{Тип документа}

Значение опции \PCBdocCode{doctype} определяет тип документа:

\begin{tabular}{cll}
  & \PCBdocCode{pe~~} & Перечень элементов\\
  & \PCBdocCode{pp~~} & Чертёж печатной платы\\
  & \PCBdocCode{sb~~} & Сборочный чертёж\\
  & \PCBdocCode{sch~} & Схема электрическая принципиальная\\
  & \PCBdocCode{spec} & Спецификация
\end{tabular}

Если опция \PCBdocCode{doctype} не указана, типом документа по
умолчанию является спецификация.

\subsection{Размер страницы}

Если типом документа является чертёж печатной платы, сборочный чертёж или схема
электрическая принципиальная, с помощью опции
\PCBdocCode{papersize} имеется возможность указать размер
страницы. Если же типом документа является перечень элементов или спецификация,
опция \PCBdocCode{papersize} игнорируется, и размер страницы
устанавливается в значение по умолчанию. Если опция
\PCBdocCode{papersize} не указана, размер страницы также
устанавливается в значение по умолчанию.

Опция \PCBdocCode{papersize} может принимать значения \PCBdocCode{a4},
\PCBdocCode{a3}, \PCBdocCode{a2}, \PCBdocCode{a1}, \PCBdocCode{a4x3} и
\PCBdocCode{a4x4}. Размером страницы по умолчанию является \PCBdocCode{a4}.

\subsection{Толщины линий}

Опции \PCBdocCode{linethick} и \PCBdocCode{linethin} задают ширину толстой и
тонкой линии соответственно. Значениями по умолчанию данных опций являются
\PCBdocCode{0.6mm} и \PCBdocCode{0.3mm} соответственно.
