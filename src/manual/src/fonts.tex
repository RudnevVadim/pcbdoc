
\section{Шрифты}

По умолчанию в \emph{pcbdoc} используется наклонный шрифт \bfemph{GOST type A},
размер которого зависит от контекста. В качестве прямого имеется возможность
использовать шрифт \bfemph{GOST Type AU}. Локально сменить наклонный шрифт на прямой
и(или) изменить размер шрифта можно с помощью команд, приведённых в
\bfemph{Таблице~\ref{tabular:font}} и \bfemph{Таблице~\ref{tabular:fontit}}.~

\begin{longtable}{%
>{\sffamily\bfseries\itshape\small}lc%
>{\ttfamily}lc%
l%
}%
\label{tabular:font}\\
\caption{Команды изменения размера прямого шрифта}\\
\hline\hline
\multicolumn{1}{c}{\sffamily\bfseries{}Команда} & &
\multicolumn{1}{c}{\sffamily\bfseries{}Пример использования} & &
\multicolumn{1}{c}{\sffamily\bfseries{}Результат}\\
\endfirsthead
\caption{Команды изменения размера прямого шрифта. Продолжение}\\
\hline\hline
\multicolumn{1}{c}{\bfseries{}Команда} & &
\multicolumn{1}{c}{\bfseries{}Пример использования} & &
\multicolumn{1}{c}{\bfseries{}Результат}\\
\endhead
\cellcolor{codecolor}\textbackslash{}small & &
\cellcolor{codecolor}%
\textcolor{Blue}{\textbackslash{}AuthorSet\{\textbackslash{}small\{\}}Пупкин%
\textcolor{Blue}{\}} & &
\cellcolor{resultcolor}\smallresult{}Пупкин\\
\cellcolor{codecolor}\textbackslash{}normalfont & &
\cellcolor{codecolor}%
\textcolor{Blue}{\textbackslash{}AuthorSet\{\textbackslash{}normalfont\{\}}Пупкин%
\textcolor{Blue}{\}} & &
\cellcolor{resultcolor}\normalfontresult{}Пупкин\\
\cellcolor{codecolor}\textbackslash{}llarge & &
\cellcolor{codecolor}%
\textcolor{Blue}{\textbackslash{}AuthorSet\{\textbackslash{}llarge\{\}}Пупкин%
\textcolor{Blue}{\}} & &
\cellcolor{resultcolor}\llargeresult{}Пупкин\\
\cellcolor{codecolor}\textbackslash{}large & &
\cellcolor{codecolor}%
\textcolor{Blue}{\textbackslash{}AuthorSet\{\textbackslash{}large\{\}}Пупкин%
\textcolor{Blue}{\}} & &
\cellcolor{resultcolor}\largeresult{}Пупкин\\
\cellcolor{codecolor}\textbackslash{}LLarge & &
\cellcolor{codecolor}%
\textcolor{Blue}{\textbackslash{}AuthorSet\{\textbackslash{}LLarge\{\}}Пупкин%
\textcolor{Blue}{\}} & &
\cellcolor{resultcolor}\LLargeresult{}Пупкин\\
\cellcolor{codecolor}\textbackslash{}Large & &
\cellcolor{codecolor}%
\textcolor{Blue}{\textbackslash{}AuthorSet\{\textbackslash{}Large\{\}}Пупкин%
\textcolor{Blue}{\}} & &
\cellcolor{resultcolor}\Largeresult{}Пупкин\\
\end{longtable}

\begin{longtable}{%
>{\sffamily\bfseries\itshape\small}lc%
>{\ttfamily}lc%
>{\ttfamily}l%
}%
\label{tabular:fontit}\\
\caption{Команды изменения размера наклонного шрифта}\\
\hline\hline
\multicolumn{1}{c}{\sffamily\bfseries{}Команда} & &
\multicolumn{1}{c}{\sffamily\bfseries{}Пример использования} & &
\multicolumn{1}{c}{\sffamily\bfseries{}Результат}\\
\endfirsthead
\caption{Команды изменения размера наклонного шрифта. Продолжение}\\
\hline\hline
\multicolumn{1}{c}{\sffamily\bfseries{}Команда} & &
\multicolumn{1}{c}{\sffamily\bfseries{}Пример использования} & &
\multicolumn{1}{c}{\sffamily\bfseries{}Результат}\\
\endhead
\cellcolor{codecolor}\textbackslash{}smallit & &
\cellcolor{codecolor}%
\textcolor{Blue}{\textbackslash{}AuthorSet\{\textbackslash{}smallit\{\}}Пупкин%
\textcolor{Blue}{\}} & &
\cellcolor{resultcolor}\smallitresult{}Пупкин\\
\cellcolor{codecolor}\textbackslash{}normalfontit & &
\cellcolor{codecolor}%
\textcolor{Blue}{\textbackslash{}AuthorSet\{\textbackslash{}normalfontit\{\}}Пупкин%
\textcolor{Blue}{\}} & &
\cellcolor{resultcolor}\normalfontitresult{}Пупкин\\
\cellcolor{codecolor}\textbackslash{}llargeit & &
\cellcolor{codecolor}%
\textcolor{Blue}{\textbackslash{}AuthorSet\{\textbackslash{}llargeit\{\}}Пупкин%
\textcolor{Blue}{\}} & &
\cellcolor{resultcolor}\llargeitresult{}Пупкин\\
\cellcolor{codecolor}\textbackslash{}largeit & &
\cellcolor{codecolor}%
\textcolor{Blue}{\textbackslash{}AuthorSet\{\textbackslash{}largeit\{\}}Пупкин%
\textcolor{Blue}{\}} & &
\cellcolor{resultcolor}\largeitresult{}Пупкин\\
\cellcolor{codecolor}\textbackslash{}LLargeit & &
\cellcolor{codecolor}%
\textcolor{Blue}{\textbackslash{}AuthorSet\{\textbackslash{}LLargeit\{\}}Пупкин%
\textcolor{Blue}{\}} & &
\cellcolor{resultcolor}\LLargeitresult{}Пупкин\\
\cellcolor{codecolor}\textbackslash{}Largeit & &
\cellcolor{codecolor}%
\textcolor{Blue}{\textbackslash{}AuthorSet\{\textbackslash{}Largeit\{\}}Пупкин%
\textcolor{Blue}{\}} & &
\cellcolor{resultcolor}\Largeitresult{}Пупкин\\
\end{longtable}

Кроме того, для указания типа шрифта и его размера можно воспользоваться встроенными
средствами \XeLaTeX{}. Например:

\pcbdocmanualcode{%
\textcolor{Blue}{\textbackslash{}AuthorSet\{\textbackslash{}fontspec[}Scale=0.68%
\textcolor{Blue}{]\{}GOST type A\textcolor{Blue}{\}\textbackslash{}itshape\{\}}Пупкин%
\textcolor{Blue}{\}}
}
