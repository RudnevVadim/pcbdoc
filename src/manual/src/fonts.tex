
\section{Шрифты}

По умолчанию в \emph{pcbdoc} используется наклонный шрифт \bfemph{GOST type A},
размер которого зависит от контекста. В качестве прямого имеется возможность
использовать шрифт \bfemph{GOST Type AU}. Локально сменить наклонный шрифт на прямой
и(или) изменить размер шрифта можно с помощью команд, приведённых в
\bfemph{Таблице~\ref{tabular:font}} и \bfemph{Таблице~\ref{tabular:fontit}}.~

\begin{tikztablex}
{\topcaption{Команды изменения размера прямого шрифта}\label{tabular:font}}
[
  draw,
  nodes={text depth=0.5mm,text height=3mm},
  column sep=3mm,
  row 1/.style={font=\sffamily\bfseries,align=center},
  column 1/.style={font=\sffamily\bfseries\itshape\small,align=left,text width=21mm},
  column 2/.style={align=left,text width=70mm},
  column 3/.style={align=left,text width=20mm},
]
{
Команда & Пример использования & Результат\\
\textbackslash{}small       & & \smallresult{}Пупкин\\
\textbackslash{}normalfont  & & \normalfontresult{}Пупкин\\
\textbackslash{}llarge      & & \llargeresult{}Пупкин\\
\textbackslash{}large       & & \largeresult{}Пупкин\\
\textbackslash{}LLarge      & & \LLargeresult{}Пупкин\\
\textbackslash{}Large       & & \Largeresult{}Пупкин\\
};
\node[anchor=west] at(tikztable-2-2.179){%
\begin{pcbdoccode1}%
\AuthorSet{\small{}Пупкин}
\end{pcbdoccode1}
};
\node[anchor=west] at(tikztable-3-2.179){%
\begin{pcbdoccode1}%
\AuthorSet{\normalfont{}Пупкин}
\end{pcbdoccode1}
};
\node[anchor=west] at(tikztable-4-2.179){%
\begin{pcbdoccode1}%
\AuthorSet{\llarge{}Пупкин}
\end{pcbdoccode1}
};
\node[anchor=west] at(tikztable-5-2.179){%
\begin{pcbdoccode1}%
\AuthorSet{\large{}Пупкин}
\end{pcbdoccode1}
};
\node[anchor=west] at(tikztable-6-2.179){%
\begin{pcbdoccode1}%
\AuthorSet{\LLarge{}Пупкин}
\end{pcbdoccode1}
};
\node[anchor=west] at(tikztable-7-2.179){%
\begin{pcbdoccode1}%
\AuthorSet{\Large{}Пупкин}
\end{pcbdoccode1}
};
\begin{scope}[on background layer]
\coordinate(left) at (tikztable-1-1.west);
\coordinate(right) at (tikztable-1-3.east);
\coordinate(top) at (tikztable-1-2.north);
\draw[line width=0.6 mm] (top-|left) -- (top-|right);
\coordinate(right) at (tikztable-1-1.east);
\coordinate(top) at (tikztable-2-1.north);
\coordinate(bottom) at (tikztable-7-1.south);
\fill[codecolor] (top-|left) rectangle (bottom-|right);
\coordinate(left) at (tikztable-2-2.west);
\coordinate(right) at (tikztable-2-2.east);
\fill[codecolor] (top-|left) rectangle (bottom-|right);
\coordinate(left) at (tikztable-2-3.west);
\coordinate(right) at (tikztable-2-3.east);
\fill[resultcolor] (top-|left) rectangle (bottom-|right);
\end{scope}
\end{tikztablex}

\begin{tikztablex}
{\topcaption{Команды изменения размера наклонного шрифта}\label{tabular:fontit}}
[
  draw,
  nodes={text depth=0.5mm,text height=3mm},
  column sep=3mm,
  row 1/.style={font=\sffamily\bfseries,align=center},
  column 1/.style={font=\sffamily\bfseries\itshape\small,align=left,text width=21mm},
  column 2/.style={align=left,text width=75mm},
  column 3/.style={align=left,text width=20mm},
]
{
Команда & Пример использования & Результат\\
\textbackslash{}smallit       & & \smallitresult{}Пупкин\\
\textbackslash{}normalfontit  & & \normalfontitresult{}Пупкин\\
\textbackslash{}llargeit      & & \llargeitresult{}Пупкин\\
\textbackslash{}largeit       & & \largeitresult{}Пупкин\\
\textbackslash{}LLargeit      & & \LLargeitresult{}Пупкин\\
\textbackslash{}Largeit       & & \Largeitresult{}Пупкин\\
};
\node[anchor=west] at(tikztable-2-2.180){%
\begin{pcbdoccode1}%
\AuthorSet{\smallit{}Пупкин}
\end{pcbdoccode1}
};
\node[anchor=west] at(tikztable-3-2.180){%
\begin{pcbdoccode1}%
\AuthorSet{\normalfontit{}Пупкин}
\end{pcbdoccode1}
};
\node[anchor=west] at(tikztable-4-2.180){%
\begin{pcbdoccode1}%
\AuthorSet{\llargeit{}Пупкин}
\end{pcbdoccode1}
};
\node[anchor=west] at(tikztable-5-2.180){%
\begin{pcbdoccode1}%
\AuthorSet{\largeit{}Пупкин}
\end{pcbdoccode1}
};
\node[anchor=west] at(tikztable-6-2.180){%
\begin{pcbdoccode1}%
\AuthorSet{\LLargeit{}Пупкин}
\end{pcbdoccode1}
};
\node[anchor=west] at(tikztable-7-2.180){%
\begin{pcbdoccode1}%
\AuthorSet{\Largeit{}Пупкин}
\end{pcbdoccode1}
};
\begin{scope}[on background layer]
\coordinate(left) at (tikztable-1-1.west);
\coordinate(right) at (tikztable-1-3.east);
\coordinate(top) at (tikztable-1-2.north);
\draw[line width=0.6 mm] (top-|left) -- (top-|right);
\coordinate(right) at (tikztable-1-1.east);
\coordinate(top) at (tikztable-2-1.north);
\coordinate(bottom) at (tikztable-7-1.south);
\fill[codecolor] (top-|left) rectangle (bottom-|right);
\coordinate(left) at (tikztable-2-2.west);
\coordinate(right) at (tikztable-2-2.east);
\fill[codecolor] (top-|left) rectangle (bottom-|right);
\coordinate(left) at (tikztable-2-3.west);
\coordinate(right) at (tikztable-2-3.east);
\fill[resultcolor] (top-|left) rectangle (bottom-|right);
\end{scope}
\end{tikztablex}

Кроме того, для указания типа шрифта и его размера можно воспользоваться встроенными
средствами \XeLaTeX{}. Например:

\begin{pcbdoccode}
\AuthorSet{\fontspec[Scale=0.68]{GOST type A}\itshape{}Пупкин}
\end{pcbdoccode}

Команда \bfemph{\textbackslash{}plusminus} печатает символ \symbol{177}. Например:

\begin{pcbdoccode}
\Element{RMC\_0805\_1\_KOM\plusminus{}5\%}{\refbox{R1}}{1}
\end{pcbdoccode}
