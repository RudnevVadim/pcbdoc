
\section{Тренинг}

Вася нарисовал схему и разработал плату замечательного устройства --- фильтра
низкой частоты первого порядка. Плата была успешно изготовлена, спаяна и
настроена. Теперь это устройство безукоризненно выполняет свои функции на радость
клиентам фирмы, где работает Вася. Да вот беда --- начальство Васи требует
оформить перечень элементов согласно ЕСКД. Немного подумав, Вася решает
использовать для этих целей \emph{pcbdoc}.

Вася открывает свой любимый текстовый редактор(кто знает, возможно это
\emph{emacs}) и приступает к работе. Он аккуратно набирает первую строчку:

\pcbdocmanualcode{%
\textcolor{Blue}{\textbackslash{}documentclass[}doctype=pe\textcolor{Blue}{]\{}%
pcbdoc\textcolor{Blue}{\}}%
}

Теперь \LaTeX{} в курсе, что ему предстоит сверстать перечень элементов.
Далее Васе требуется указать, что он является автором этого замечательного
документа:

\pcbdocmanualcode{%
\textcolor{Blue}{\textbackslash{}AuthorSet\{}Пупкин\textcolor{Blue}{\}}
}

Готовый перечень элементов Вася планирует показать для проверки коллеге, который,
как и Вася, разрабатывает замечательные электронные устройства, рабочее место
которого находится неподалёку. Ну в самом деле, а вдруг в документ закрадётся
ошибка? Одна голова хорошо, а две лучше, думает Вася, набирая следующую
строчку:

\pcbdocmanualcode{%
\textcolor{Blue}{\textbackslash{}CheckerSet\{}Ближайший\textcolor{Blue}{\}}
}

Вася знает, что проверять готовый документ также будет и очень суровый
нормоконтролёр. Нужно указать и этот факт. Быстро и решительно Вася набирает
очередную строчку:

\pcbdocmanualcode{%
\textcolor{Blue}{\textbackslash{}NormControllerSet\{}Суровый\textcolor{Blue}{\}}
}

Вася знает, что утверждать перечень элементов будет Васин начальник. Вася очень
любит своего начальника, потому что он обещал повысить Васе зарплату. Жизнь
прекрасна, думает Вася, набирая следующую строчку:

\pcbdocmanualcode{%
\textcolor{Blue}{\textbackslash{}ApproverSet\{}Сказочник\textcolor{Blue}{\}}
}

Далее Вася указывает имя своего устройства и его децимальный номер, который
по секрету сообщил Васе его начальник отдела:

\pcbdocmanualcode{%
\textcolor{Blue}{\textbackslash{}NameSet\{}Фильтр\textcolor{Blue}{\}}\\
\textcolor{Blue}{\textbackslash{}NumberSet\{}РОГА.123456.001\textcolor{Blue}{\}}
}

Пора приступать к заполнению тела документа, думает Вася, набирая следующие
две строчки:

\pcbdocmanualcode{%
\textcolor{Blue}{\textbackslash{}begin\{}document\textcolor{Blue}{\}}\\
\textcolor{Blue}{\textbackslash{}begin\{}ElementList\textcolor{Blue}{\}}
}

Несмотря на то, что конденсатор в Васином устройстве всего один, Вася, вспомнив
про суровость нормоконтролёра,  решил создать для одного конденсатора целый
раздел:

\pcbdocmanualcode{%
\textcolor{Blue}{\textbackslash{}Part\{}Конденсаторы\textcolor{Blue}{\}}
}

Вот он, этот единственный конденсатор:

\pcbdocmanualcode{%
\textcolor{Blue}{\textbackslash{}Element\{}X7R\_0805\_1\_МКФ\_5\%%
\textcolor{Blue}{\}\{\textbackslash{}refbox\{}C1\textcolor{Blue}{\}\}\{}1%
\textcolor{Blue}{\}}%
}

Как видим, Вася не забыл указать его позиционное обозначение и
количество. Аналогично Вася поступил и с единственным резистором:

\pcbdocmanualcode{%
\textcolor{Blue}{\textbackslash{}Part\{}Резисторы\textcolor{Blue}{\}}\\
\textcolor{Blue}{\textbackslash{}Element\{}RMC\_0805\_1\_KOM\_5\%\textcolor{Blue}{\}%
\{\textbackslash{}refbox\{}R1\textcolor{Blue}{\}\}\{}1\textcolor{Blue}{\}}%
}

А вот соединителей в Васином устройстве целых два. Вася не забыл и про них:

\pcbdocmanualcode{%
\textcolor{Blue}{\textbackslash{}Part\{}Соединители\textcolor{Blue}{\}}\\
\textcolor{Blue}{\textbackslash{}Element\{}Розетка SMA-BJ\textcolor{Blue}{\}%
\{\textbackslash{}refbox\{}XS1,XS2\textcolor{Blue}{\}\}\{}2\textcolor{Blue}{\}}%
}

Ну вроде всё, подумал Вася. Пора закругляться.

\pcbdocmanualcode{%
\textcolor{Blue}{\textbackslash{}end\{}ElementList\textcolor{Blue}{\}}\\
\textcolor{Blue}{\textbackslash{}end\{}document\textcolor{Blue}{\}}
}

И сохранил полученный файл, присвоив ему имя \bfemph{vasia.tex}.

Открыв эмулятор терминала, Вася перешёл в директорию, куда он сохранил свой
файл, и скомпилировал его:

\pcbdocmanualterminal{%
cd vasia\\
xelatex vasia.tex
}

После компиляции Вася обнаружил в своей рабочей директории файл
\bfemph{vasia.pdf}. Ну надо же, подумал Вася, получилось. Но что это за странные
вопросительные знаки в графе \bfemph{Листов}? Наверное, я что-то не то сделал,
подумал Вася, повторно запуская компиляцию:

\pcbdocmanualterminal{%
xelatex vasia.tex
}

После которой странные знаки исчезли. То-то же, подумал Вася и приступил к
настройке принтера. Распечатав полученный документ,

\begin{center}
\fcolorbox{black}{resultcolor}{%
%\begin{center}
  \includegraphics[scale=0.55]{output/vasia}
%\end{center}
}
\end{center}

Вася направился собирать подписи.
