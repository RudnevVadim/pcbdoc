
\section{Перечень элементов}

Исходный код перечня элементов, как и любого другого документа \LaTeX{}, должен
начинаться с команды \bfemph{\textbackslash{}documentclass}. В основном аргументе
команды(в фигурных скобках) следует указать класс документа \sfemph{pcbdoc}, а в
необязательном(в квадратных скобках) --- ключ \sfemph{doctype=pe}.

\begin{pcbdoccode}
\documentclass[doctype=pe]{pcbdoc}
\end{pcbdoccode}

Далее(в преамбуле) должны находиться команды заполнения полей документа. Их можно
разместить либо непосредственно в преамбуле, либо в отдельном пакете, включаемом в
преамбулу с помощью команды \bfsf{\textbackslash{}usepackage}. Тело документа,
находящееся после преамбулы, должно начинаться с команды

\begin{pcbdoccode}
\begin{document}
\end{pcbdoccode}

а заканчиваться командой

\begin{pcbdoccode}
\end{document}
\end{pcbdoccode}

%Внутри тела документа должно находиться
%окружение \sfemph{ElementList}, которое начинается с команды
%
%\begin{pcbdoccode}
%\begin{ElementList}
%\end{pcbdoccode}
%
%а заканчивается командой
%
%\begin{pcbdoccode}
%\end{ElementList}
%\end{pcbdoccode}

Внутри тела документа должны находиться команды заполнения строк
перечня элементов, приведённые в \bfsf{Таблице~\ref{tabular:pelines}}.\newpage

\tikztableset{my table}{
  draw,
  row 1/.style={minimum height=6mm,font=\sffamily\bfseries,align=center},
  column 1/.style={font=\sffamily\bfseries\itshape\small,align=left,text width=39mm},
  column 2/.style={font=\small,align=justify,text width=101mm},
  row 3/.style={minimum height=8mm},
  row 5/.style={minimum height=66mm}
}

\begin{tikztablex}[my table]
{\caption{Команды заполнения строк перечня элементов}\label{tabular:pelines}}
{
Команда & Описание\\
\textbackslash{}Part\{<name>\} &
Печатает подчёркнутый аргумент \sfemph{<name>} в центре колонки
\resultbox{\sfemph{Наименование}} Например:\\
&\\
{\textbackslash{}Element[<note>]\\
\{<naming>\}\\
\{<refdes1 ... refdesN>\}\\
\{<quantity>\}} &
Заполняет строку перечня элементов. Необязательный аргумент \sfemph{<note>} печатается
в колонке \resultbox{\sfemph{Примечание}}. Аргументы \sfemph{<naming>},
\sfemph{<refdes1 ... refdesN>} и \sfemph{<quantity>} печатаются в колонках
\resultbox{\sfemph{Наименование}},
\resultbox{\sfemph{Поз. обозначение}} и
\resultbox{\sfemph{Кол.}} соответственно.
Каждая запись в аргументе \sfemph{<refdes1 ... refdesN>} печатается в отдельной строке
колонки \resultbox{\sfemph{Поз. обозначение}}. Каждую строку в колонке
\resultbox{\sfemph{Поз. обозначение}} следует отделять от предыдущей одним
или несколькими пробельными символами(символ перехода на другую строку также является
пробельным символом). Между символом \bfsf{\{} и первой строкой, а также между
последней строкой и символом \bfsf{\}} пробельных символов быть не должно, поэтому в
примере ниже используется символ подавления последующих пробельных символов
\bfsf{\%}. Каждую запись в строке позиционных обозначений рекомендуется размещать в
аргументе команды \bfsf{\textbackslash{}refbox}, которая центрирует запись внутри
колонки. Шрифт любой записи строки позиционных обозначений можно немного
уменьшить(например, с помощью команды \bfsf{\textbackslash{}llargeit}), что позволит
разместить в колонке чуть выступающий за её пределы текст. Например:\\
&\\
\textbackslash\textbackslash & Переход на новую строку \\
\textbackslash{}newsheet & Переход на новую страницу \\
};
\node[right=40mm,anchor=center] at ($(tikztable-3-1)+(0,0)$){
\begin{pcbdoccode1}
\Part{Микросхемы}
\end{pcbdoccode1}
};
\node[right=53mm,anchor=center] at ($(tikztable-5-1)+(0,0)$){
\begin{pcbdoccode1}
\Element{Y5V\_1206\_4,7\_MKF\_20\%\_25V}{%
  \refbox{C6,C15,C16}
  \refbox{C18,C21}
  \refbox{C174,C175}
  \refbox{C180,C181}
  \refbox{C184-C187}
  \refbox{C190,C191}
  \refbox{C195-C199}
  \refbox{C201,C204}
  \refbox{C205}
  \refbox{\llargeit{}C207-C210}
  \refbox{C212}
  \refbox{\llargeit{}C232-C234}
  \refbox{\llargeit{}C238-C240}
  \refbox{\llargeit{}C265-C266}%
  }{36}
\end{pcbdoccode1}
};
\coordinate(top) at (tikztable-1-2.north);
\coordinate(left) at (tikztable-1-1.west);
\coordinate(right) at (tikztable-1-2.east);
\draw[line width=0.6mm] (top-|left) -- (top-|right);
\coordinate(top) at (tikztable-1-2.south);
\draw[line width=0.6mm] (top-|left) -- (top-|right);

\coordinate(top) at (tikztable-4-2.north);
\draw (top-|left) -- (top-|right);

\coordinate(top) at (tikztable-6-2.north);
\draw (top-|left) -- (top-|right);

\coordinate(top) at (tikztable-7-2.north);
\draw (top-|left) -- (top-|right);

\coordinate(top) at (tikztable-7-2.south);
\draw[line width=0.6mm] (top-|left) -- (top-|right);

\begin{scope}[on background layer]
\coordinate(top) at (tikztable-2-2.north);
\coordinate(bottom) at (tikztable-2-2.south);
\coordinate(left) at (tikztable-1-1.west);
\coordinate(right) at (tikztable-1-1.east);
\fill[codecolor] (top-|left) rectangle (bottom-|right);
\coordinate(top) at (tikztable-4-2.north);
\coordinate(bottom) at (tikztable-4-2.south);
\fill[codecolor] (top-|left) rectangle (bottom-|right);
\coordinate(top) at (tikztable-6-2.north);
\coordinate(bottom) at (tikztable-7-2.south);
\fill[codecolor] (top-|left) rectangle (bottom-|right);
\fill[codecolor]
  ($(tikztable-3-1.north west)-(0,1.5mm)$)
  rectangle
  ($(tikztable-3-2.south east)+(0,1.5mm)$);
\fill[codecolor]
  ($(tikztable-5-1.north west)-(0,1.5mm)$)
  rectangle
  ($(tikztable-5-2.south east)+(0,1.5mm)$);
\end{scope}
\end{tikztablex}
