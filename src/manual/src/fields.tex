
\section{Команды заполнения полей документа}

Команды заполнения полей, приведённые в таблице~\ref{tabular:fields}, должны
находиться либо непосредственно в преамбуле документа, либо в отдельном пакете,
включаемом в преамбулу с помощью команды \coloremph{\textbackslash{}usepackage}.

\begin{longtable}{%
>{\ttfamily\bfseries}p{0.26\textwidth}%
>{\small}p{0.68\textwidth}%
}%
\label{tabular:fields}\\
\caption{Команды заполнения полей документа}\\
\hline\hline
\multicolumn{1}{c}{\sffamily\bfseries{}Команда} &
\multicolumn{1}{c}{\sffamily\bfseries{}Описание}\\
\hline\hline
\endfirsthead
\caption{Команды заполнения полей документа. Продолжение}\\
\hline\hline
\multicolumn{1}{c}{\sffamily\bfseries{}Команда} &
\multicolumn{1}{c}{\sffamily\bfseries{}Описание}\\
\hline\hline
\endhead
\cellcolor{codecolor}\textbackslash{}AuthorSet \{\coloremph{<name>}\} &
Печатает аргумент \coloremph{<name>} в поле \coloremph{Разраб.} основной надписи.\\
\hline
\cellcolor{codecolor}\textbackslash{}CheckerSet \{\coloremph{<name>}\} &
Печатает аргумент \coloremph{<name>} в поле \coloremph{Пров.} основной надписи.\\
\hline
\cellcolor{codecolor}\textbackslash{}ScaleSet \{\coloremph{<value>}\} &
Печатает аргумент \coloremph{<value>} в поле \coloremph{Масштаб} основной надписи
сборочного чертежа или чертежа печатной платы.\\
\hline
\cellcolor{codecolor}\textbackslash{}NormControllerSet \{\coloremph{<name>}\} &
Печатает аргумент \coloremph{<name>} в поле \coloremph{Н.~контр.} основной надписи.\\
\hline
\cellcolor{codecolor}\textbackslash{}TechControllerSet \{\coloremph{<name>}\} &
Печатает аргумент \coloremph{<name>} в поле \coloremph{Т.~контр.} основной надписи
сборочного чертежа или чертежа печатной платы.\\
\hline
\cellcolor{codecolor}\textbackslash{}ApproverSet \{\coloremph{<name>}\} &
Печатает аргумент \coloremph{<name>} в поле \coloremph{Утв.} основной надписи.\\
\hline
\cellcolor{codecolor}\textbackslash{}NameSet \{\coloremph{<name>}\} &
Печатает аргумент \coloremph{<name>} в поле наименования изделия основной надписи.
Аргумент \coloremph{<name>} может быть как однострочным, так и двустрочным. Разделение
аргумента на строки производится с помощью команды
\texttt{\bfseries\textbackslash\textbackslash}. Например:
\makebox[0.68\textwidth]{%
\pcbdocmanualcode{%
  \textbackslash{}NameSet\{Модуль\textbackslash\textbackslash{}расширителя сознания\}%
  }%
}\\
\hline
\cellcolor{codecolor}\textbackslash{}NumberSet \{\coloremph{<number>}\} &
Печатает децимальный номер \coloremph{<number>} в поле обозначения документа основной
надписи схемы электрической принципиальной, перечня элементов, сборочного чертежа и
спецификации, а также в поле \coloremph{Перв. примен.} схемы электрической
принципиальной, перечня элементов, чертежа печатной платы и сборочного чертежа.\\
\hline
\cellcolor{codecolor}\textbackslash{}PcbNumberSet \{\coloremph{<number>}\} &
Печатает децимальный номер \coloremph{<number>} в поле обозначения документа основной
надписи чертежа печатной платы. \\
\hline
\cellcolor{codecolor}\textbackslash{}PcbMaterialSet \{\coloremph{<name>}\} &
Печатает аргумент \coloremph{<name>} в поле обозначения материала детали основной
надписи чертежа печатной платы. Аргумент \coloremph{<name>} может быть однострочным,
двустрочным или трёхстрочным. Разделение аргумента на строки производится с помощью
команды \texttt{\bfseries\textbackslash\textbackslash}. Например:
\makebox[0.68\textwidth]{%
\pcbdocmanualcode{%
  \textbackslash{}PcbMaterialSet\{Материал фольгированный%
  \textbackslash\textbackslash\\качественный\textbackslash\textbackslash{}от надёжного
  поставщика%
  \}%
  }%
}\\
\hline
\cellcolor{codecolor}\textbackslash{}PrimaryUseSet \{\coloremph{<number>}\} &
Печатает аргумент \coloremph{<number>} в поле \coloremph{Перв. примен.} основной
надписи спецификации.\\
\hline\hline
\end{longtable}
