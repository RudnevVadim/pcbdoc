
\section{Команды заполнения полей документа}

Команды заполнения полей, приведённые в \bfemph{Таблице~\ref{tabular:fields}},
должны находиться либо непосредственно в преамбуле документа, либо в отдельном
пакете, включаемом в преамбулу с помощью команды
\bfemph{\textbackslash{}usepackage}.

\tikztableset{my table}{
  draw,
  row 1/.style={font=\sffamily\bfseries,align=center},
  column 1/.style
    ={font=\sffamily\bfseries\itshape\small,align=left,text width=37.6mm},
  column 2/.style={font=\small,align=justify,text width=102.6mm},
  row 5/.style={minimum height=10mm},
  row 9/.style={minimum height=8mm},
  row 13/.style={minimum height=12mm}
}

\begin{tikztablex}[my table]
{\caption{Команды заполнения полей документа}\label{tabular:fields}}
{
  Команда & Описание\\
  \textbackslash{}AuthorSet\{<name>\} &
  Печатает аргумент \sfemph{<name>} в поле \resultbox{\sfemph{Разраб.}}
  основной надписи.\\
  \textbackslash{}CheckerSet\{<name>\} &
  Печатает аргумент \sfemph{<name>} в поле \resultbox{\sfemph{Пров.}}
  основной надписи.\\
  \textbackslash{}ScaleSet\{<value>\} &
  Печатает аргумент \sfemph{<value>} в поле \resultbox{\sfemph{Масштаб}}
  основной надписи сборочного чертежа или чертежа печатной платы.\\
  {\textbackslash{}NormControllerSet\\\{<name>\}} &
  Печатает аргумент \sfemph{<name>} в поле \resultbox{\sfemph{Н.~контр.}}
  основной надписи.\\
  {\textbackslash{}TechControllerSet\\\{<name>\}} &
  Печатает аргумент \sfemph{<name>} в поле \resultbox{\sfemph{Т.~контр.}}
  основной надписи сборочного чертежа или чертежа печатной платы.\\
  \textbackslash{}ApproverSet\{<name>\} &
  Печатает аргумент \sfemph{<name>} в поле \resultbox{\sfemph{Утв.}}
  основной надписи.\\
  \textbackslash{}NameSet\{<name>\} &
  Печатает аргумент \sfemph{<name>} в поле наименования изделия
  основной надписи. Аргумент \sfemph{<name>} может быть как
  однострочным, так и двустрочным. Разделение аргумента на строки производится с
  помощью команды \bfemph{\textbackslash\textbackslash}. Например:\\
  &\\
  \textbackslash{}NumberSet\{<number>\} &
  Печатает децимальный номер \sfemph{<number>} в поле обозначения
  документа основной надписи схемы электрической принципиальной, перечня
  элементов, сборочного чертежа и спецификации, а также в поле
  \resultbox{\sfemph{Перв. примен.}} схемы электрической принципиальной,
  перечня элементов, чертежа печатной платы и сборочного чертежа.\\
  {\textbackslash{}PcbNumberSet\\\{<number>\}} &
  Печатает децимальный номер \sfemph{<number>} в поле обозначения
  документа основной надписи чертежа печатной платы.\\
  {\textbackslash{}PcbMaterialSet\\\{<name>\}} &
  Печатает аргумент \sfemph{<name>} в поле обозначения материала
  детали основной надписи чертежа печатной платы, а также сборочного чертежа.
  Аргумент \sfemph{<name>} может быть однострочным, двустрочным или
  трёхстрочным. Разделение аргумента на строки производится с помощью команды
  \bfemph{\textbackslash\textbackslash}. Например:\\
  &\\
  \textbackslash{}PrimaryUseSet \{<number>\} &
  Печатает аргумент \sfemph{<number>} в поле
  \resultbox{\sfemph{Перв. примен.}} основной надписи спецификации.\\
};
\coordinate(left) at (tikztable-1-1.west);
\coordinate(right) at (tikztable-1-2.east);
\draw[line width=0.6mm]
  coordinate(top) at (tikztable-1-2.north) (top-|left) -- (top-|right);
\draw[line width=0.6mm]
  coordinate(top) at (tikztable-2-2.north) (top-|left) -- (top-|right);
\foreach \x in {3,...,8}{
\draw coordinate(top) at (tikztable-\x-2.north) (top-|left) -- (top-|right);
}
\foreach \x in {10,...,12}{
\draw coordinate(top) at (tikztable-\x-2.north) (top-|left) -- (top-|right);
}
\draw coordinate(top) at (tikztable-14-2.north) (top-|left) -- (top-|right);
\draw[line width=0.6mm]
  coordinate(top) at (tikztable-14-2.south) (top-|left) -- (top-|right);
\node[right=30mm,anchor=center] at ($(tikztable-9-1) + (0,0)$){
\begin{pcbdoccode1}
\NameSet{Модуль\\расширителя сознания}
\end{pcbdoccode1}
};
\node[right=30mm,anchor=center] at ($(tikztable-13-1) + (0,0)$){
\begin{pcbdoccode1}
  \PcbMaterialSet{Материал фольгированный\\
  качественный\\от надёжного поставщика}
\end{pcbdoccode1}
};
\begin{scope}[on background layer]
\coordinate(right) at (tikztable-1-1.east);
\coordinate(top) at (tikztable-2-2.north);
\coordinate(bottom) at (tikztable-8-2.south);
\fill[codecolor] (top-|left) rectangle (bottom-|right);
\coordinate(top) at (tikztable-10-2.north);
\coordinate(bottom) at (tikztable-12-2.south);
\fill[codecolor] (top-|left) rectangle (bottom-|right);
\coordinate(top) at (tikztable-14-2.north);
\coordinate(bottom) at (tikztable-14-2.south);
\fill[codecolor] (top-|left) rectangle (bottom-|right);
\fill[codecolor]
  ($(tikztable-9-1.north west)-(0,1.5mm)$)
  rectangle
  ($(tikztable-9-2.south east)+(0,1.5mm)$);
\fill[codecolor]
  ($(tikztable-13-1.north west)-(0,1.5mm)$)
  rectangle
  ($(tikztable-13-2.south east)+(0,1.5mm)$);
\end{scope}
\end{tikztablex}

\clearpage

\tikztableset{my table}{
  draw,
  row 1/.style={font=\sffamily\bfseries,align=center},
  column 1/.style
    ={font=\sffamily\bfseries\itshape\small,align=left,text width=39.6mm},
  column 2/.style={font=\small,align=justify,text width=100.6mm},
  row 5/.style={minimum height=10mm},
  row 9/.style={minimum height=8mm},
  row 13/.style={minimum height=12mm}
}

\begin{tikztablex}[my table]
{
  \caption*{Таблица~\ref{tabular:fields}.Команды заполнения полей
  документа. Продолжение}
}
{
  Команда & Описание\\
  \textbackslash{}CompanySet\{<name>\} &
  Печатает аргумент \sfemph{<name>} в поле наименования или различительного
  индекса предприятия. Аргумент \sfemph{<name>} может быть как однострочным,
  так и двустрочным.\\
  \textbackslash{}ClientSignSet\{<name>\} &
  При включённой опции \bfemph{extstamp} печатает аргумент \sfemph{<name>} в поле
  знака, установленного заказчиком в соответствии с требованиями нормативной
  документации.\\
  {\textbackslash{}LiterSolutionNumberSet\\\{<name>\}} &
  При включённой опции \bfemph{extstamp} печатает аргумент \sfemph{<name>} в поле
  номера решения и года утверждения соответствующей литеры.\\
  {\textbackslash{}SolutionNumberSet\\\{<name>\}} &
  При включённой опции \bfemph{extstamp} печатает аргумент \sfemph{<name>} в поле
  номера решения и года утверждения документации.\\
  {\textbackslash{}ClientIndexSet\\\{<name>\}} &
  При включённой опции \bfemph{extstamp} печатает аргумент \sfemph{<name>} в поле
  индекса заказчика в соответствии с нормативной документацией.\\
};
\coordinate(left) at (tikztable-1-1.west);
\coordinate(right) at (tikztable-1-2.east);
\draw[line width=0.6mm]
  coordinate(top) at (tikztable-1-2.north) (top-|left) -- (top-|right);
\draw[line width=0.6mm]
  coordinate(top) at (tikztable-2-2.north) (top-|left) -- (top-|right);
\draw[line width=0.6mm]
  coordinate(top) at (tikztable-6-2.south) (top-|left) -- (top-|right);
\begin{scope}[on background layer]
\coordinate(right) at (tikztable-1-1.east);
\coordinate(top) at (tikztable-2-2.north);
\coordinate(bottom) at (tikztable-6-2.south);
\fill[codecolor] (top-|left) rectangle (bottom-|right);
\end{scope}
\end{tikztablex}

\clearpage
