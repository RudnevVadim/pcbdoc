
\section{Команды заполнения полей документа}

Команды заполнения полей, приведённые в \bfemph{Таблице~\ref{tabular:fields}}, должны
находиться либо непосредственно в преамбуле документа, либо в отдельном пакете,
включаемом в преамбулу с помощью команды \bfemph{\textbackslash{}usepackage}.\\[-15mm]
\begin{longtable}{%
>{\ttfamily\bfseries}p{0.26\textwidth}%
>{\small}p{0.68\textwidth}%
}%
\label{tabular:fields}\\
\caption{Команды заполнения полей документа}\\
\hline\hline
\multicolumn{1}{c}{\sffamily\bfseries{}Команда} &
\multicolumn{1}{c}{\sffamily\bfseries{}Описание}\\
\hline\hline
\endfirsthead
\caption{Команды заполнения полей документа. Продолжение}\\
\hline\hline
\multicolumn{1}{c}{\sffamily\bfseries{}Команда} &
\multicolumn{1}{c}{\sffamily\bfseries{}Описание}\\
\hline\hline
\endhead
\cellcolor{codecolor}%
\textbackslash{}AuthorSet\{\sfemph{<name>}\} &
Печатает аргумент \sfemph{<name>} в поле \sfemph{Разраб.} основной
надписи.\\
\hline
\cellcolor{codecolor}%
\textbackslash{}CheckerSet\{\sfemph{<name>}\} &
Печатает аргумент \sfemph{<name>} в поле \sfemph{Пров.} основной
надписи.\\
\hline
\cellcolor{codecolor}

\vspace{-4mm}
\textbackslash{}ScaleSet\{\sfemph{<value>}\} &
Печатает аргумент \sfemph{<value>} в поле \sfemph{Масштаб} основной
надписи сборочного чертежа или чертежа печатной платы.\\
\hline
\cellcolor{codecolor}%
\textbackslash{}NormControllerSet \{\sfemph{<name>}\} &
Печатает аргумент \sfemph{<name>} в поле \sfemph{Н.~контр.} основной
надписи.\\
\hline
\cellcolor{codecolor}%
\textbackslash{}TechControllerSet \{\sfemph{<name>}\} &
Печатает аргумент \sfemph{<name>} в поле \sfemph{Т.~контр.} основной
надписи сборочного чертежа или чертежа печатной платы.\\
\hline
\cellcolor{codecolor}%
\textbackslash{}ApproverSet \{\sfemph{<name>}\} &
Печатает аргумент \sfemph{<name>} в поле \sfemph{Утв.} основной
надписи.\\
\hline
\cellcolor{codecolor}

\vspace{1mm}
\textbackslash{}NameSet\{\sfemph{<name>}\} &
Печатает аргумент \sfemph{<name>} в поле наименования изделия
основной надписи. Аргумент \sfemph{<name>} может быть как
однострочным, так и двустрочным. Разделение аргумента на строки производится с помощью
команды \bfemph{\textbackslash\textbackslash}. Например:\\\\[-4mm]
\multicolumn{2}{c}%
{\pcbdocmanualcode%
{\textbackslash{}NameSet\{Модуль\textbackslash\textbackslash{}расширителя сознания\}%
}}\\
\hline
\cellcolor{codecolor}

\vspace{3mm}
\textbackslash{}NumberSet \{\sfemph{<number>}\} &
Печатает децимальный номер \sfemph{<number>} в поле обозначения
документа основной надписи схемы электрической принципиальной, перечня элементов,
сборочного чертежа и спецификации, а также в поле \sfemph{Перв. примен.} схемы
электрической принципиальной, перечня элементов, чертежа печатной платы и сборочного
чертежа.\\
\hline
\cellcolor{codecolor}%
\textbackslash{}PcbNumberSet \{\sfemph{<number>}\} &
Печатает децимальный номер \sfemph{<number>} в поле обозначения
документа основной надписи чертежа печатной платы.\\
\hline
\cellcolor{codecolor}

\vspace{1mm}
\textbackslash{}PcbMaterialSet \{\sfemph{<name>}\} &
Печатает аргумент \sfemph{<name>} в поле обозначения материала
детали основной надписи чертежа печатной платы. Аргумент
\sfemph{<name>} может быть однострочным, двустрочным или
трёхстрочным. Разделение аргумента на строки производится с помощью команды
\bfemph{\textbackslash\textbackslash}. Например:\\\\[-4mm]
%\hline\\[-4mm]
\multicolumn{2}{c}%
{\pcbdocmanualcode%
{\textbackslash{}PcbMaterialSet\{Материал фольгированный%
\textbackslash\textbackslash\\качественный\textbackslash\textbackslash{}от надёжного
поставщика\}}}\\
\hline
\cellcolor{codecolor}%
\textbackslash{}PrimaryUseSet \{\sfemph{<number>}\} &
Печатает аргумент \sfemph{<number>} в поле \sfemph{Перв. примен.}
основной надписи спецификации.\\
\hline\hline
\end{longtable}
