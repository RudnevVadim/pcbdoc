
\section{Шрифты}

В \emph{pcbdoc} используется наклонный шрифт \emph{GOST type A}(по умолчанию) и
прямой \emph{GOST Type AU}.

Предусмотрены команды изменения шрифта. Наклонный шрифт, от меньшего к
большему:

\pcbdocmanualcode{%
\textbackslash{}smallit\\
\textbackslash{}normalfontit\\
\textbackslash{}llargeit\\
\textbackslash{}largeit\\
\textbackslash{}LLargeit\\
\textbackslash{}Largeit
}

Прямой шрифт, от меньшего к большему:

\pcbdocmanualcode{%
\textbackslash{}small\\
\textbackslash{}normalfont\\
\textbackslash{}llarge\\
\textbackslash{}large\\
\textbackslash{}LLarge\\
\textbackslash{}Large
}

Пример применения:

\pcbdocmanualcode{%
\textbackslash{}AuthorSet\{\textbackslash{}smallit\{\}Пупкин\}
}

Кроме того, Вы можете указать тип шрифта и его размер "в лоб" средствами
\XeLaTeX{}. Например:

\pcbdocmanualcode{%
\textbackslash{}AuthorSet\{\textbackslash{}fontspec[Scale=0.68]\{GOST type A\}%
\textbackslash{}itshape\{\}Пупкин\}
}

И, само собой разумеется, если Вы планируете часто использовать конкретный размер
шрифта, имеет смысл создать для этого новую команду.
